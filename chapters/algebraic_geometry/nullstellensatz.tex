\section{The Nullstellensatz}
The key philosophy of algebraic geometry is to understand geometry in an algebraic way. Let $\mathbb{A}^n = k^n$, where $k$ is an algebraically closed field, such as $\mathbb{C}$. A key obeservation is:
$$\begin{aligned}
    \text{Point }(a_1,\cdots,a_n) \in \mathbb{A}^n &\Longleftrightarrow \mathfrak{m}_a  \in \MaxSpec k[x_1,\cdots,x_n], \\
    \text{Algebraic set }V(I) \subset \mathbb{A}^n &\Longleftrightarrow \text{Radical ideals } \sqrt{I}.
\end{aligned}$$
This correspondence, known as the \textit{Nullstellensatz}, is the origin of algebraic geometry. In this language, we can find a very elegant correspondence between geometrical properties and algebraic properties.
